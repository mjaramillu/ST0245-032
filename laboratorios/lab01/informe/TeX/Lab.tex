\documentclass[a4paper,12pt]{article}
\usepackage[hmargin=2cm,top=4cm,headheight=65pt,footskip=45pt]{geometry}
\usepackage[utf8]{inputenc}
\usepackage{graphicx}
\usepackage[hidelinks]{hyperref}
\usepackage{array}
\usepackage{lastpage}
\usepackage{lipsum}
\usepackage{fancyvrb}
\usepackage{color}
\usepackage{fancyhdr}
\usepackage{amsmath}
\usepackage{enumitem}
\usepackage{titlesec}

\definecolor{customGray}{RGB}{128,128,128}
%\definecolor{Eblue}{RGB}{0,21,87}
%\definecolor{Eblue}{rgb}{0.2, 0.2, 0.6}
\definecolor{Eblue}{rgb}{0.0, 0.18, 0.39}
%==============Header & Footnote==============

\pagestyle{fancy}
\renewcommand{\headrulewidth}{0pt}
\fancyhead[C,CO,L,LO,R,RO]{}
\fancyhead[C]{%
          \begin{tabular}{|m{3.0cm}|m{10.0cm}|m{2.5cm}|}
          \hline
          \centering\vspace{1.75mm}\includegraphics[scale=0.32]{logo.png} &
          \centering
          {\footnotesize {\sf UNIVERSIDAD EAFIT\\ SCHOOL OF ENGINEERING\\
          \vspace{-1mm}DEPARTMENT OF SYSTEMS AND INFORMATICS}} &
          \centering
          \footnotesize{Page \thepage\ de \pageref{LastPage}\\
          ST245\\
          \vspace{-0.75mm}Data Structures
          }\tabularnewline
          \hline
          \end{tabular}%
}
\fancyfoot[C,CO,L,LO,R,RO]{}
\fancyfoot[C]{
          \begin{centering}
            \textcolor{customGray}{{\footnotesize {\sf Professor Mauricio Toro Bermudez\\
            Phone: $(+57) (4) 261 95 00$ Ext. $9473$. Office: $19 - 627$\\
            \vspace{-1mm}E-mail: mtorobe@eafit.edu.co}}}
        \end{centering}
}

%=============CustomEnumItem===========

\setlist[enumerate]{label=\color{Eblue}\textbf{\roman*.}}

%=============CustomSecSubSec==========

\titleformat{\section}[hang]
{\normalsize\bfseries\itshape\color{black}}{\bfseries\itshape\color{Eblue}\thesection)}{2.5mm}{}

\titleformat{\subsection}[hang]
{\normalsize\bfseries\itshape\color{black}}{\bfseries\color{Eblue}\thesection.\alph{subsection}.}{2.5mm}{}

%==============Title==============

\title{\color{Eblue}\textbf{Laboratory practice No. 1: Recursion}}
\author{
  \textbf{Juan P. Ossa Zapata}\\
  Universidad EAFIT\\
  Medellín, Colombia\\
  jpossaz@eafit.edu.co
\and
  \textbf{Mauricio Jaramillo Uparela}\\
  Universidad EAFIT\\
  Medellín, Colombia\\
  mjaramillu@eafit.edu.co
}

%=============Document=============
\begin{document}
  \maketitle
  \thispagestyle{fancy}

  \section{ONLINE EXERCISES (CODINGBAT)}
  \subsection{Recursion I}
    \begin{enumerate}
      \item \begin{Verbatim}
      public int countPairs(String str) {             // c0 * n
        if (str.length() == 2 || str.length() == 1    // c1 * n
          || str.length() == 0) {                     // c1 * n
          return 0;                                   // c2 * n
        } else if (str.charAt(0) == str.charAt(2)) {  // c2 * n
          return 1 + countPairs(str.substring(1));    // c2 * T(n-1)
        } else {                                      // c3
          return countPairs(str.substring(1));        // c3 * T(n-1)
        }
      }
      \end{Verbatim}
      \item \begin{Verbatim}
      public int countHi2(String str) {               // c0 * n
        if (str.length() == 1 || str.length() == 0) { // c1 * n
          return 0;                                   // c1 * n
        } else if (str.charAt(0) == 'x') {            // c2 * n
          if (str.charAt(1) == 'h'
          && str.charAt(2) == 'i') {                  // c2 * c3 * n
            return countHi2(str.substring(2));        // c2 * c3 * T(n-2)
          } else {                                    // c2 * c4 * T(n-1)
            return countHi2(str.substring(1));        // c2 * c4 * T(n-1)
          }
        } else if (str.charAt(0) == 'h'
          && str.charAt(1) == 'i') {                  // c5 * n
          return 1 + countHi2(str.substring(1));      // c5 * T(n-1)
        } else {                                      // c6 * n
          return countHi2(str.substring(1));          // c6 * T(n-1)
        }
      }
      \end{Verbatim}
      \item \begin{Verbatim}
      public int countAbc(String str) {               // c0 * n
        if (str.length() == 0 || str.length() == 1
        || str.length() == 2) {                       // c1 * n
          return 0;                                   // c1 * n
        } else if (str.charAt(0) == 'a'
          && str.charAt(1) == 'b'
          && (str.charAt(2) == 'c'
          || str.charAt(2) == 'a')) {                 // c2 * n
          return 1 + countAbc(str.substring(1));      // c2 * T(n-1)
        } else {                                      // c3 * n
          return countAbc(str.substring(1));          // c3 * T(n-1)
        }
      }
      \end{Verbatim}
      \item \begin{Verbatim}
      public String parenBit(String str) {
        if (str.length() == 0 || str.length() == 1) {
          return "";
        } else if (str.charAt(0) == '(') {
          int count = 0;
          while (str.charAt(count) != ')') {
            count++;
          }
          count++;
          return str.substring(0, count) + parenBit(str.substring(count));
        } else {
          return parenBit(str.substring(1));
        }
      }
      \end{Verbatim}
      \item \begin{Verbatim}
      public int strCount(String str, String sub) {
        if (str.length() == 0) {
          return 0;
        } else {
          int i = 0;
          while (i < sub.length()) {
            if (sub.charAt(i) == str.charAt(i)) {
              i++;
            } else {
              break;
            }
          }
          if (i == sub.length()) {
            return 1 + strCount(str.substring(i), sub);
          } else {
            return strCount(str.substring(1), sub);
          }
        }
      }
      \end{Verbatim}
    \end{enumerate}
    \subsection{Recursion II}
    \begin{enumerate}
      \item \begin{Verbatim}
      public boolean splitArray(int[] nums) {
        return splitArrayAux(nums, 0, 0, 0);
      }
      public boolean splitArrayAux(int [] nums, int start,
        int first, int second) {
        if (start == nums.length) {
          return first == second;
        } else {
          return splitArrayAux(nums, start + 1,
            first + nums[start], second)
          || splitArrayAux(nums, start + 1, first,
            second + nums[start]);
        }
      }
      \end{Verbatim}
      \item \begin{Verbatim}
      public boolean splitOdd10(int[] nums) {
        return splitOdd10Aux(nums, 0, 0, 0);
      }
      public boolean splitOdd10Aux(int [] nums, int start,
        int first, int second) {
        if (start == nums.length) {
          return (first % 10 == 0) && (second % 2 != 0);
        } else {
          return splitOdd10Aux(nums, start + 1,
            first + nums[start], second) ||
          splitOdd10Aux(nums, start + 1,
            first, second + nums[start]);
        }
      }
      \end{Verbatim}
      \item \begin{Verbatim}
      public boolean groupSumClump(int start, int[] nums, int target) {
        if (start >= nums.length) {
          return target == 0;
        }
        int sum = 0;
        int i;
        for (i = start; i < nums.length; i++) {
          if (nums[i] == nums[start]){
            sum += nums[start];
          } else {
            break;
          }
        }
        return groupSumClump(i, nums, target - sum)
        || groupSumClump(i, nums, target);
      }
      \end{Verbatim}
      \item \begin{Verbatim}
      public boolean groupSum5(int start, int[] nums, int target) {
        if (start == nums.length) {
          return target == 0;
        } else {
          if (nums[start] % 5 == 0) {
            return groupSum5(start + 1, nums, target - nums[start]);
          } else if (start > 0 && nums[start] == 1
            && nums[start - 1] % 5 == 0) {
            return groupSum5(start + 1, nums, target);
          } else {
            return groupSum5(start + 1, nums, target - nums[start])
            || groupSum5(start + 1, nums, target);
          }
        }
      }
      \end{Verbatim}
      \item \begin{Verbatim}
      public boolean split53(int[] nums) {
        return split53Aux(nums, 0, 0, 0);
      }
      public boolean split53Aux(int [] nums, int start,
        int first, int second) {
        if (start == nums.length) {
          return first == second;
        } else {
          if (nums[start] % 5 == 0) {
            return split53Aux(nums, start + 1, first + nums[start], second);
          } else if (nums[start] % 3 == 0) {
            return split53Aux(nums, start + 1, first, second + nums[start]);
          } else {
            return split53Aux(nums, start + 1, first + nums[start], second)
            || split53Aux(nums, start + 1, first, second + nums[start]);
          }
        }
      }
      \end{Verbatim}
    \end{enumerate}
    \section{What did you learn about \texttt{Stack Overflow}?}
      The Stack Overflow error is caused by a bad recursive call -for example you do not make
      the problem simpler every time you make a recursive call- or when you do not have a stopping
      condition. In Java,
\end{document}
